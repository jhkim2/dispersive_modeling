\documentclass[11pt]{article}
\usepackage{epsfig,epsf,rotating,graphics,colordvi,color,a4}
\textwidth=16cm
\textheight=22cm
\begin{document}
\newcommand{\refpoint}[1]{\ \vspace{0.3cm}\\ {\em #1}\  \vspace{0.3cm}\\ }
\newcommand{\refit}[1]{\em #1}
\newcommand{\refitem}[1]{\item {\em #1} }
\newcommand{\todo}[1]{\ \\ {\bf To do: #1}\\}
\section*{Reviewer \#1:} 
\refpoint{The paper has been appreciably improved upon revision. However some further amend-
ments are needed to achieve a contribution worthy of publication.}
We gladly recognize that the changes and extensions which have followed from the referees suggestions 
in the first revision have improved the paper crucially. We are truly grateful for this. 
\refpoint{The main issues to be resolved are with presentation. In particular, dimensionless
variables are said to be given as starred variables (see the beginning of page 5). However,
in most of sections 3 and 4 this rule is forgotten (see also “Specific Issues”). This should
be fixed.}
Yes, indeed, we are struggling with dimensions. We prefer the Briggs experiments with dimensions and, hence, we shift a little back and forth.
To mitigate the problem we specify what is used where more thoroughly.
\begin{itemize}
\item We elaborate on scaling in the start of section 2
\item We use the star for non-dimensional quantities in text and figures. However, in some figures we instead prefer a brief descriptive text on the axes. 
\item It is specified in the beginning of some of the subsection, where we have just switched, wether dimensions are used or not.
\item Likewise, we specify in the captions of some figures.
\item Also the use of $A/h$ was a little untidy. We introduce a normalized mesure, $\alpha$, for
the amplitude of the incident wave, and leave $A/h$ for the local normalized amplitude.
\end{itemize}
    
\section*{Specific issues}
\begin{enumerate}
\refitem{ Abstract. Where appropriate, e.g. in the Abstract, reference to the model not being
fully nonlinear should be added;}\\
We agree. It now stated ``On the other hand, in the selected formulation only some non-linearity is retained in
the dispersion term. ......
Even though the equations of BoussClaw are not fully nonlinear they perform far better
than standard Boussinesq equations with only linear dispersion terms.''
 \refitem{ page 5, lines 2-3. Mention should be added that dimensionless (or normalized vari-
ables) are starred;}
Yes. This is done. In addition the whole issue is somewhat elaborated in the revision.
\refitem{ page 5, line 5. $u^* = \sqrt{gh_0}$ is a velocity scale, rather than a normalized velocity.
Please, amend;}\\
It is  done.
\refitem{ page 6, lines 2-5. The definition of $w$ (vertical velocity, I guess) is missing. Please,
amend;}\\
No, $w$ is not the vertical velocity, but a dummy variable employed to define the operator.
This is now explicitly stated in the text.
\refitem{ page 10, lines 1-13. I believe that the threshold depth is $d = 10^{-4}$m. If so, please,
amend;
Also be aware of theoretical and numerical issues related with the use of frictional-
type terms at the shoreline (e.g. Antuono et al., 2012);}
The $d$ was imprecise/wrong. The correct expression is slightly different from the one suggested by
the referee, namely $d=10^{-4}h_0$.\\
Concerning, Antuono et al: Both this reference and a very brief discussion of the effect of bottom friction on 
the withdrawal were omitted by mistake. It is now in place at the end of section 3.3.2.

\refitem{ page 12, last paragraph. Are energies here dimensional or dimensionless? In the
former case units should be given in plots, in the latter case variables should be
starred;}\\
Yes, the energies are dimensionless. For the relative errors in the right panel this does not matter, but
for the left panel it does. The caption and the text of section 3.1 is changed. 
\refitem{ figures 6, 7, 8, 9, 10, 11, 12, 13, 14, 15 and related text. Dimensionless variables
(elevation, space, time, etc.) should be starred;}\\
We have put stars on the quantities in the figures. We also often states the use of dimensionless 
quantities  in the captions. In the text we have changed to starred quantities.
\refitem{ page 23, lines 7-8. The meaning of this sentence is unclear to me. Please, clarify;}\\
Sorry. The text was not well formed. Hopefully, it is clearer now. 
\refitem{ page 25, lines 9-10. If I read well figure 15, $E_0$ of BoussClaw remains almost constant,
its increase being 3 − 4\% of the value attained using NLSW. Hence, I would not
speak of a “significant increase”;}\\
Agreed. We have changed to ``slightly, but noticeably,''
\refitem{ page 32, second half of the page. It is here said that the energy of a wave can be
approximated as $e = (e0+e1+O(\mu^4))$. However, inspection of definitions (B.2) and
(B.3) suggests that $e$ is reduced by the “equilibrium energy” $\rho gh^2/2$, which does
not appear in (B.2) and (B.3). Please, clarify;}\\
This and the next point touch upon an error in the manuscript. The expressions as they were are not correct in case of runup or drawdown. The appendix is 
amended to correct this and make the definitions clearer. Also figures 15 and 16 are modified, even though the changes are small since these figures display the energies during shoaling.
\refitem{ page 33, last line of Appendix B.2. What does “$x_a$ and $x_b$ denote the limitations of
the wetted region” mean? Please, give clear definitions of $x_a$ and $x_b$.}\\
The integral for the total energy is modified and the 
limits, including a new one, are more explicitly explained.
\end{enumerate}
\bibliographystyle{plain}  % typesettingsformatet
\bibliography{mybibfile.bib}
\end{document}


