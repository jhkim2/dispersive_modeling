\begin{align}
(I-D^2)(u_t + u_x) +D^2u_x = 0, \label{eq:basic_2}
\end{align}
where $D=\partial_x$.
Then we are required to solve the following PDE,
\begin{align}
\left\{
\begin{array}{l}
u_t + u_x + S = 0, \\
\left(I-D^2 \right)S = D^3 u .
\end{array}
\right.
\end{align}
When the hybrid scheme is applied, the advection equation
\begin{align}
u_t + u_x = 0,
\label{eq:append_advec}
\end{align}
is solved with the finite volume method, 
and then the fractional step method is applied to
the following equation,
\begin{align*}
u_t + S = 0.
\end{align*}

If we use the centered difference approximation of $O(\Delta x^2)$
accuracy, and the four-stage Runge-Kutta scheme for time stepping,
then we have,
\begin{align*}
U_1 &:= u^n \\
U_2 &:= u^n - \frac{\Delta t}{2}S_1, \quad  \textrm{where}\quad 
(I-D^2)S_1 = D^3 U_1, \\
U_3 &:= u^n - \frac{\Delta t}{2}S_2, \quad  \textrm{where}\quad
(I-D^2)S_2 = D^3 U_2, \\
U_4 &:= u^n - \Delta tS_3, \quad  \textrm{where}\quad
(I-D^2)S_3 = D^3 U_3, \\
u^{n+1} & = u^n -\frac{\Delta t}{6} \left[
S_1 + 2S_2 + 2S_3 + S_4
\right],  \quad  \textrm{where}\quad
(I-D^2)S_4 = D^3 U_4.
\end{align*}
For each step, $S_i$'s are computed from
\begin{align*}
& (S_i)_j - \frac{(S_i)_{j+1}-2(S_i)_j+(S_i)_{j-1}}{\Delta x^2} = 
\frac{u_{j+2} - 2u_{j+1} +2u_{j-1} -u_{j-2}}{2\Delta x^3}.
\end{align*}
In order to investigate the stability with von Neumann analysis,
replace $u_j= e^{i\xi j \Delta x}$ and  $(S_1)_j= \beta e^{i\xi j \Delta x}$. 
Simplification gives
\begin{align*}
&\beta \left( 1 - \frac{-2+2\cos(\xi \Delta x) }{\Delta x^2} \right) = 
\frac{\sin(2\xi \Delta x) - 2 \sin(\xi \Delta x)}{\Delta x^3} i, \\
& \beta = \frac{ -2\sin(\xi \Delta x)(1-  \cos(\xi \Delta x)) }
                     { \Delta x^3 +2\Delta x(1-\cos(\xi \Delta x))} i.
\end{align*}
Note that $\beta$ is a purely imaginary number.
If we replace the four-stage Runge-Kutta scheme with
$u_j= e^{i\xi j \Delta x}$ and  $(S_1)_j= \beta e^{i\xi j \Delta x}$, then we have
\begin{align*}
&S_1 = \beta e^{i\xi j \Delta x} ,
 \quad U_2 = \left(1-\frac{\Delta t\beta}{2} \right) e^{i\xi j \Delta x}\\
&S_2 = \beta \left(1-\frac{\Delta t\beta}{2} \right) e^{i\xi j \Delta x} ,
 \quad U_3 = \left(1- \frac{\Delta t\beta}{2}+\frac{(\Delta t\beta)^2}{4} \right) e^{i\xi j \Delta x}\\
&S_3 =  \beta\left(1- \frac{\Delta t\beta}{2}+\frac{(\Delta t\beta)^2}{4} \right)  e^{i\xi j \Delta x} , \\
& U_4 = \left(1-\Delta t\beta
+ \frac{(\Delta t\beta)^2}{2}-\frac{(\Delta t\beta)^3}{4} \right) e^{i\xi j \Delta x},\\
&S_4 = \beta \left(1-\Delta t\beta
+ \frac{(\Delta t\beta)^2}{2}-\frac{(\Delta t\beta)^3}{4} \right) e^{i\xi j \Delta x}.
\end{align*}
Thus the growth factor $g(\xi)$ is
\begin{align*}
g(\xi)= & 1-\frac{\Delta t}{6}\bigg[ \beta + 
2 \beta \left(1-\frac{\Delta t\beta}{2} \right) +
2 \beta\left(1- \frac{\Delta t\beta}{2}+\frac{(\Delta t\beta)^2}{4} \right) \\ 
&+ \beta \left(1-\Delta t\beta
+ \frac{(\Delta t\beta)^2}{2}-\frac{(\Delta t\beta)^3}{4} \right)
\bigg] \\
= & 1-\frac{1}{6}\left(
6\Delta t\beta -3(\Delta t\beta)^2 +(\Delta t\beta)^3-\frac{(\Delta t\beta)^4}{4}
\right).
\end{align*}
Since $\beta$ is an imaginary number,
let $\Delta t\beta= \gamma i$ for some real $\gamma$, and then we have
\begin{align*}
g(\gamma) & = 1-
\frac{1}{2}\gamma^2 +\frac{\gamma^4}{24} + \left(\frac{\gamma^3}{6} -\gamma \right)i \\
|g(\gamma)|^2 & = 1 + \frac{1}{4}\gamma^4 + \frac{\gamma^8}{24^2} -\gamma^2 + \frac{\gamma^4}{12}
-\frac{\gamma^6}{24} + \gamma^2 + \frac{\gamma^6}{36} -\frac{\gamma^4}{3} \\
& = 1 -\frac{1}{72}\gamma^6 + \frac{1}{576}\gamma^8.
\end{align*}
If $|\gamma|<2\sqrt{2}$, then $|g(\gamma)|<1$. 
The sufficient condition for stability is 
\begin{align}
\left| \frac{\Delta t}{\Delta x} \cdot \frac{ \sin(\xi \Delta x)(1-  \cos(\xi \Delta x)) }
                     { \Delta x^2 +2(1-\cos(\xi \Delta x))} \right| < \sqrt{2}, 
                     \quad \textrm{for~} \forall \xi \Delta x.
\end{align}
For small $\Delta x$, this condition approximately reduces to
\[
\frac{\Delta t}{\Delta x} < 2\sqrt{2}.
\]
The CFL condition for the advection equation
(\ref{eq:append_advec}) is a sufficient condition. 
Therefore, if the CFL condition is satisfied in the advection equation,
the fractional step is always stable with the suggested numerical scheme. 
\begin{align}
\left| \frac{\Delta t}{\Delta x} \cdot \frac{ \sin(\xi \Delta x)(1-  \cos(\xi \Delta x)) }
                     { \Delta x^2 +2(1-\cos(\xi \Delta x))} \right| < \sqrt{2}, 
                     \quad \textrm{for~} \forall \xi \Delta x.
\end{align}
For small $\Delta x$, this condition approximately reduces to

\section{Comparison of the Boussinesq equations}
\label{append:a}

A class of the Boussinesq-type equations can be written 
in the following form,
\begin{flalign}
& (H)_t + (Hu)_x = 0, \\
& (Hu)_t + \left( Hu^2 + \frac{gH²}{2} \right)_x + gHh_x + \psi = 0,
\end{flalign}
where $\psi$ represents dispersion terms.
If 
\begin{flalign}
\psi = \frac{Hh^2}{6} u_{xxt} - \frac{Hh}{2} (hu)_{xxt},
\label{eq:peregrine_disp}
\end{flalign}
then these are called 
the Peregrine's equations.

We claim that the Boussinesq equations 
by \citet{schaffer1993boussinesq}
can be approximately reduced to the Peregrine's form when $B=0$.
Since $H=h+\eta$, the dispersion terms of Sch{\"a}ffer and Madsen
with $B=0$, can be written as, 
\begin{flalign}
\psi = & \frac{1}{6}h^3 \left(\frac{Hu}{h} \right)_{xxt}
- \frac{1}{2}h^2 (Hu)_{xxt} \nonumber \\
= & \frac{1}{6}h^3 u_{xxt}
+ \frac{1}{6}h^3 \left( \frac{\eta u}{h} \right)_{xxt}
- \frac{1}{2}h^2 (hu)_{xxt} - \frac{1}{2}h^2 (\eta u)_{xxt} \nonumber \\
= & \frac{H h^2}{6} u_{xxt} - \frac{H h}{2} (hu)_{xxt} \nonumber \\
&- \frac{\eta h^2}{6} u_{xxt}
+ \frac{h^3}{6} \left( \frac{\eta u}{h} \right)_{xxt}
+ \frac{\eta h}{2} (hu)_{xxt}
- \frac{h^2}{2} (\eta u)_{xxt} \label{eq:append_schaffer_disp1} \\
= & \frac{H h^2}{6} u_{xxt} - \frac{H h}{2} (hu)_{xxt}
+ \mathcal{O}(\epsilon). \nonumber
\end{flalign}
Because the last four terms in (\ref{eq:append_schaffer_disp1})
are $\mathcal{O}(\epsilon$),
the dispersion terms of Sch{\"a}ffer and Madsen are approximately same as
the Peregrine's. 
However, due to the higher order terms in (\ref{eq:append_schaffer_disp1}),
Sch{\"a}ffer and Madsen's wave model has lower peak 
near the wave break
than the Peregrine's model. 

Now we will study the connection between 
Sch{\"a}ffer and Madsen's equations 
and Serre's equations. 
Use $H=h+\eta$ and $\eta_t = H_t = -(Hu)_x$,
and assume $\eta_{xx}$ is small. 
The dispersion terms of Sch{\"a}ffer and Madsen 
can be rewritten in a different form, 
\begin{flalign*}
\psi = & \frac{1}{6}h^3 \left(\frac{Hu}{h} \right)_{xxt}
- \frac{1}{2}h^2 (Hu)_{xxt} \\
= & - Hh h_xu_{xt} -\frac{1}{3} H h^2 u_{xxt}- H h \eta_xu_{xt} \\
& - \frac{1}{6} Hh \eta u_{xxt} 
+ \frac{1}{6} h \eta^2 u_{xxt} + h \eta (h+\eta)_x u_{xt} 
+ \frac{1}{6} h^3 \left( \frac{\eta u}{h} \right)_{xxt}
\end{flalign*}
Meanwhile
\begin{flalign*}
\left( \frac{\eta u}{h} \right)_{xxt} 
= & -\frac{2 h_x}{h^2}\eta u_{xt} 
+ \frac{2}{h} \eta_x u_{xt}
+ \frac{1}{h} \eta u_{xxt} 
+ \frac{2 h_x}{h^2} (Hu)_x u_{x} \\
&- \frac{2}{h} (Hu)_{xx} u_x 
- \frac{1}{h} (Hu)_x u_{xx}.
\end{flalign*}
Therefore $\psi$ reduces to 
\begin{flalign*}
\psi 
= & Hh\eta_xu_{xt} + \left( \frac{2}{3} h_x \eta  
- \frac{2}{3} h \eta_x  \right) hu_{xt} \\
& + \frac{1}{6} h^3 \left(\frac{2 h_x}{h^2} (Hu)_x u_{x}
- \frac{2}{h} (Hu)_{xx} u_x 
- \frac{1}{h} (Hu)_x u_{xx}  \right)  \\
= & - Hh h_xu_{xt} -\frac{1}{3} H h^2 u_{xxt}
+ \left( \frac{2}{3} h_x \eta  
- \frac{2}{3} h \eta_x  \right) hu_{xt} \\
& + \frac{1}{6} h^3 \left(\frac{2 h_x}{h^2} (Hu)_x u_{x}
- \frac{2}{h} (Hu)_{xx} u_x 
- \frac{1}{h} (Hu)_x u_{xx}  \right)
\end{flalign*}
Now consider the last term 
\begin{flalign*}
 & \frac{1}{6} h^3 \left(\frac{2 h_x}{h^2} (Hu)_x u_{x}
- \frac{2}{h} (Hu)_{xx} u_x 
- \frac{1}{h} (Hu)_x u_{xx}  \right) \\
= & \frac{1}{3} h \left( h_x (Hu)_x u_{x}
- h (Hu)_{xx} u_x 
- \frac{h}{2} (Hu)_x u_{xx}  \right) \\
= & \frac{1}{3} h \left( h_x \left( H_xu + Hu_x \right) u_{x}
- h \left( 2H_xu_x + Hu_{xx} \right) u_x 
- \frac{h}{2} (H_xu + Hu_x) u_{xx}  \right) \\
= & \frac{1}{3} h \left( h_x H (u_x)^2 
- 2H_x h (u_x)^2 - H h u_x u_{xx} 
- \frac{h}{2} H_x u u_{xx} - \frac{h}{2} H u_x u_{xx} \right) \\
= & \frac{1}{3} h \left( \left( h_x H - 2H_x h \right) (u_x)^2 
- \frac{3 H h}{2} u_x u_{xx} 
- \frac{h}{2} H_x u u_{xx} \right)
\end{flalign*}
If we rearrange (\ref{eq:madsen_new_disp_x}) and
assume that $h_x\eta$, $h_{xx}$ and $\eta_{xx}$ are small,
then 
the Madsen's dispersion terms can be written as
\begin{flalign*}
\psi 
\approx & - Hh h_xu_{xt} -\frac{Hh^2}{3} u_{xxt}- H h \eta_xu_{xt}
- \frac{1}{6} Hh \eta u_{xxt} 
+ \frac{1}{6} h^3 \left( \frac{\eta u}{h} \right)_{xxt}.
\end{flalign*}
Since $\eta_t = -(Hu)_x$, we have
\begin{flalign*}
\left( \frac{\eta u}{h} \right)_{xxt} 
\approx & \frac{2 \eta_x}{h} u_{xt}
+ \frac{\eta}{h} u_{xxt}  &+ \frac{2 h_x}{h^2} (Hu)_x u_{x}
- \frac{2}{h} (Hu)_{xx} u_x 
- \frac{1}{h} (Hu)_x u_{xx}.
\end{flalign*}
Plugging in and dropping small terms yields to 
\begin{flalign}
\psi 
\approx & - Hh h_xu_{xt} -\frac{Hh^2}{3} u_{xxt}- H h \eta_xu_{xt} 
- \frac{1}{6} Hh \eta u_{xxt} \nonumber \\
& +\frac{h^2\eta_x}{3} u_{xt}
+ \frac{h^2\eta}{6} u_{xxt} + \frac{h h_x}{3} (Hu)_x u_{x}
- \frac{h^2}{3} (Hu)_{xx} u_x 
- \frac{h^2}{6} (Hu)_x u_{xx} \nonumber \\
\approx & - Hh h_xu_{xt} -\frac{Hh^2}{3} u_{xxt}- H h \eta_xu_{xt} \nonumber \\
& +\frac{h^2\eta_x}{3} u_{xt}
 + \frac{h h_x}{3} (Hu)_x u_{x}
- \frac{h^2}{3} (Hu)_{xx} u_x 
- \frac{h^2}{6} (Hu)_x u_{xx}. \label{eq:madsen_disp_v3}
\end{flalign}
%\marginpar{\small I am not sure how to conclude the relation between Boussinesq and Serre.}

The Serre's equations of \citep{su1969korteweg}
has the following dispersion terms,
\begin{flalign}
\psi= &-Hhh_xu_{xt} - \frac{Hh^2}{3}u_{xxt}
- Hh\eta_xu_{xt} \nonumber \\
& - \frac{2}{3} Hh\eta u_{xxt}
+ \frac{Hh^2}{3}\left[ (u_x)^2-uu_{xx} \right]_x.
\label{eq:serre_disp}
\end{flalign}
It is observed that (\ref{eq:madsen_disp_v3}) and (\ref{eq:serre_disp}) share common terms.
As shown in Figure \ref{fig:bim_boussclaw_fun}
numerical results also show that Sch{\"a}ffer and Madsen's 
model is more similar to Serre's equations than Peregrine's equations.


