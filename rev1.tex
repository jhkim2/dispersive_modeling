\documentclass[11pt]{article}
\usepackage{epsfig,epsf,rotating,graphics,colordvi,color,a4}
\textwidth=16cm
\textheight=22cm
\begin{document}
\newcommand{\refpoint}[1]{\ \vspace{0.3cm}\\ {\em #1}\  \vspace{0.3cm}\\ }
\newcommand{\refit}[1]{\em #1}
\newcommand{\todo}[1]{\ \\ {\bf To do: #1}\\}
\section*{Reviewer \#1:} 
We appreciate the useful suggestions of the referee and realize that much of his criticism is 
well founded. On the other hand, even though it to  some extent reflects weaknesses in our 
submission, we cannot quite accept his negative assessments of novelty and focus of our paper 
and his general dismissal of much of its content.  
Concerning the alleged lack of focus, for instance, it must be accepted to address more than 
a single subject in a paper. While the first submission did not present the double focus with sufficient clarity, it was
still explicitly stated in the introduction. We hope that the revised version will do this better.  As a part of the revision we have reorganized
the paper with this in mind. Main changes: 
\begin{enumerate}
\item The introduction is reorganized. More references are added, some very detailed parts are omitted.
\item Most features of the models are now all collected in section 2
\item A part of the old section 4 is moved to 3.3. There it is followed by two  new tests sections with runup.
\item The ill-named section 4 is no longer a section, one part is moved to section 3, another to the new section 4.1 (old 5.1). 
\item The old section 5, which the new section 4, is changed with new figures and amended text.
      In particular the energy part is revised with dissipation rates and more discussion.
\item The number of figures on pre-breaking shoaling is reduced. 
\end{enumerate}
Below we respond to the points raised by the referee. For the longer, general comments we have taken the liberty of
introducing en enumeration to particular objections and questions.

\section*{Main issues}
\refpoint{
{\bf Innovation content.} This is very unclear and it is a fundamental task of the Authors
to highlight the actual innovation of their contribution. Such innovation does not
seem to reside in the modeling of tsunami waves by means of a Boussinesq model:
the literature reports already a large amount of papers on this. Further, more novel
approaches are becoming available (see also point 3 of “Specific Issues”). Neither
the numerical technique in itself is an innovation: many NLSW and Boussinesq
models use the “fractional step method” or “operatorial splitting method”$\,^1$. Not
even the benchmarking approach is new$^2$: many examples are available of contrasting
a Boussinesq model with a fully nonlinear potential flow solver (see, a.o., Wei et al.,
1995)$\,^3$. Thus, is the aim of this paper that of proposing a model more accurate than
those already available?$\,^4$  Or that of studying the breaking criterion?$\,^5$ None of these
is focussed upon in the work, neither detailed in the title, which is very general and
somehow uninformative. A significant effort should be made to clarify the actual
innovation of the present contribution$\,^6$;}
\begin{enumerate}
\item To be fair, most ``new'' numerical methods are only a modification of existing models or a combination
        of well known numerical techniques. This is particularly true for the field of Boussinesq models. 
      We present our special combination of tools and procedures with the final aim of making available a free 
      alternative to FunWave/CoulWave, within the ClawPack software. As far as we have seen this mix is innovative
      in the sense that we have not found something identical in the literature. But, then the literature on Boussinesq
      equations is vast and while equations and mathematical details are presented lavishly, the papers are
      still incomplete and tests and simulations are difficult to assess properly. Our experience so far indicates 
     that the  presented approach, for the applications we consider, is much more promising than the alternatives 
     from the literature that we have tested ourselves, namely the FunWave/CoulWave. 
\item The application of potential theory was over-emphasized in the first submission.
      This is now corrected. Naturally, comparison with potential theory as such
      is not new. In fact,  we have compared with full potential theory ourselves, when feasible, 
      much as a routine (examples: \cite{Lovholt:2013a,Pedersen:2013}).
      When it is done properly,  we think that  it strengthens investigations like
      the present one and contribute to important insight. In fact,  most paper on long wave models do not
      invoke a ``full model'', but a number of these papers would really have benefited from doing so.
\item This specific reference \cite{wei1995fully} is well spotted by the referee and 
      we are to blame for not making use of it. The reference is well known and parts of the content, 
      in particular figure 4, are closely related to some of our work and must thus be explicitly and properly referenced.
       However, the reference applies different
      equations from us (in particular we have non-fully nonlinear Boussinesq equations that perform well), the slope
      of \cite{synolakis1987runup}, which we employ, is strangely enough not included and nothing is presented
      after the potential model yield breaking. Hence, we must do this investigation ourselves, but it is played down in the revised 
version and  \cite{wei1995fully} is briefly discussed.  
\item The final aim is to provide a general purpose model, within ClawPack, that is reliable and stable, rather than formally 
more accurate than other models. After all, it is not fully nonlinear.  We have now stated this  in the introduction.
\item Yes, that as well. Section 4 is much changed in the revised manuscript. We will also stress that the two
      topics are closely related. We could not have addressed the second without access to a suitable model.
      We have not seen the discussion of, for instance, the $\epsilon_B=0.8$ threshold, done is this manner before.
      Hence, it is novel and, in our view, also important.      
\item The title is changed.
\end{enumerate}
\refpoint{ {\bf Model Benchmarking.} All benchmarking is here made with reference to the free 
surface elevation evolution. However, for any purpose of practical interest, like that of
studying dispersive tsunamis, it is essential to assess the value of the model in 
representing also the internal kinematics, e.g. the reproduction of the vertical distribution
of the flow$^1$. Further, the Abstract seems to suggest some interest in the accurate
modeling of inundation. However, no specific test is here proposed to assess the
model description of inundation, references to inundation and wetting-drying only
appearing in the Introduction and Conclusions$^2$.}
\begin{enumerate}
\item Really, the statement ``for any purpose...'' is a little too strong in our opinion. Putting that aside:\\
We have made some graphs on vertical distribution of velocities, but in our view they did not contribute to the paper.
Of course we could compare velocity profiles constructed from the Boussinesq models to BIM results, as was done in   \cite{wei1995fully}, but
this would only inflate the pre-breaking part which we seek to play down a bit. If we, for instance, had presented new Boussinesq type 
equations of higher 
order this  would have been different.
\item Good point. Tests of runup performance of the model is now included.   
\end{enumerate}
\section*{Specific issues}
{\em The following specific issues require attention (line numbers are those provided by the
authors’ editing):}
Unfortunately, here we have a problem. These line numbers are provided by the processing of the journal.
This processed version appears  no longer to be available at the WEB site. We have to guess then. 
\begin{enumerate}
\item{\em Title. As already mentioned, the title is too general and uninformative. Once the
Authors have decided the actual focus of their contribution, the title should be
changed to reflect it;}\\
We have changed it. 
\item{\em Abstract, lines 22-24. What here stated is actually incorrect, proof is that an
increasing number of studies is being dedicated to solve also dispersive dynamics of
tsunamis;}\\
Here we are at an loss concerning which statement that offends the referee. The closest we 
get is the two first lines, but they state the same as the referee....
\item {\em Introduction. The Introduction contrasts NLSW and Boussinesq models as tools
used for describing the evolution of tsunami waves. However, it is becoming clear
that dispersive tsunamis can be modeled by a variety of novel models, which deserve,
at least, some account in this introductory overview.\\
Among them:}
\begin{itemize}
\item[{\em (a)}] {\em the conceptually innovative Dispersive NLSW of Antuono et al. (2009), which
combine the capability of reproducing frequency dispersion with the properties
of hyperbolic systems;}
\item[{\em (b)}] {\em the number of multi-layer and non-hydrostatic models, a summary of which
can be found in the recent review on Boussinesq models of Brocchini (2013);}
\end{itemize}
We do not make a full review in our introduction. The referenced review papers of Brocchini, which was left out 
by mistake, Kirby (new) and Madsen do this far better anyway.
There are a number of papers that may ``deserve'' mentioning, which we  still not refer to  as the primary references
always are those which we draw on directly.  The paper Antuono et al. (2009) is quite interesting and
a bit different from most others, even though it is not that central in relation to our work. Hence, it is included briefly in the introduction, but not discussed in detail.    
\item{\em page 4, lines 21-23. This sentence is unclear and needs rewording;}\\
The whole page only contains 24 text lines. Assuming that its double counting due to
the space line. We end in the first paragraph which, indeed, was not well formulated. 
The start of section 2 is anyway completely rewritten.  
\item{\em page 4, lines 35-39. This sentence is unclear and needs rewording;}\\
Probably, here as well the wording is now changed.
\item{\em Section 3. It would be very beneficial to add quantitative estimates (measures of
errors) to illustrations for all comparative analyses;}\\
Yes. We already had a plot of the energy errors in solitary wave propagation.
We have now included graphs and tables with maximum runup heights (dependence on grid and mathematical formulation) for the tests in section 3.3. We believe this should suffice.  
\item{\em Figure 10 and related text. The solution by BoussClaw displays a significantly larger
energy content at the front face of the wave with respect to all other solutions. Can
the Authors explain the origin of such a discrepancy?}\\
The BoussClaw wave front is ahead of all the others. The Serre model has a little of the same
behaviour, while the other models show the opposite in relation to the BIM solution, which is
the ``correct'' one. It has to do with omitted nonlinearities in the dispersion term, then. Any deeper or firmer
explanation is difficult to establish. In addition to displaying the discrepancies in the figures, we now explicitly state them in the text.
\item{\em Figure 13 and related text. The energy decay rate for NLSW displayed by the $E_0$
signal in figure (13) could be compared with the energy actually dissipated by the
solver across a bore through the Rankine-Hugoniot conditions}\\
Good suggestion. Such a comparison is now included.
The whole section on energies (old 5.2, new 4.3) has also been rewritten.
\end{enumerate}
\section*{Editing}
$\bullet${\em  page 2, line 44. ...“could reach”... instead of ...“could reaches”...;}\\
This part of the sentence was not informative anyway and is deleted.
\bibliographystyle{plain}  % typesettingsformatet
\bibliography{mybibfile.bib}
\end{document}


